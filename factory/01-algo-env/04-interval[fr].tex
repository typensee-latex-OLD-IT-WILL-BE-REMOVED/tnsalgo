\documentclass[12pt,a4paper]{article}

\makeatletter
	\input{../config/header[fr].sty}
	
	\usepackage{04-interval}
\makeatother


\begin{document}

%\section{Algorithmes en language naturel}

\subsection{Intervalles discrets d'entiers}

Il est d'usage en informatique théorique de poser $\CSinterval{4}{7} = \{ 4 ; 5 ; 6 ; 7 \}$ où \CSinterval{4}{7} a été obtenu en tapant \verb#\CSinterval{4}{7}#.
Le préfixe \prefix{CS} fait référence à \prefix{Computer Science} soit \inenglish{Informatique Théorique}.


% ---------------------- %


\subsection{Fiches techniques}

\IDmacro[a]{CSinterval}{2}   \hfill \prefix{CS = C-omputer S-cience}

\IDarg{1} la borne entière inférieure.

\IDarg{2} la borne entière supérieure.

\end{document}

