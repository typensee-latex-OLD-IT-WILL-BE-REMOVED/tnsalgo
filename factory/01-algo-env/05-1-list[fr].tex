\documentclass[12pt,a4paper]{article}

\makeatletter
	\input{../config/header[fr].sty}
	\usepackage{00-keywords}
	\usepackage{01-algo-env}
	\usepackage{03-affectation}
	
	\usepackage{05-list}
\makeatother


\begin{document}

%\section{Des outils pour les algorithmes} \label{algo-extra}

\subsection{Listes}

Avant de commencer il faut savoir que la convention retenue pour la numérotation des indices des listes est de commencer à $1$ \emph{(ceci est plus naturel pour un humain)}.


\subsubsection{Opérations de base.}

\newparaexample{Longueur d'une liste}

\begin{latexex}
$\Len(L)$
\end{latexex}


% ---------------------- %


\newparaexample{Liste vide et liste en extension}

\begin{latexex}
$\EmptyList$

$\List{4 ; 7 ; -1}$
\end{latexex}


% ---------------------- %


\newparaexample{Extraire certains éléments}

\begin{latexex}
k\ieme{} élément :
$\ListElt{L}{k}$

Du 1\ier{} au k\ieme{} :
$\ListUntil{L}{k}$

À partir du k\ieme{} :
$\ListFrom{L}{k}$
\end{latexex}


% ---------------------- %


\newparaexample{Concaténer des listes}

\begin{latexex}
$L =          \ListUntil{L}{k-1}
     \AddList \ListElt{L}{k}
     \AddList \ListFrom{L}{k+1}$
\end{latexex}


% ---------------------- %


\subsubsection{Modifier une liste avec un élément}

\newparaexample{Versions textuelles}

\prefix{append} et \prefix{prepend} signifient \inenglish*{ajouter} et \inenglish{préfixer}.
Quant à \prefix{pop}, il signifie \inenglish*{éclater}.


\begin{latexex}
Ex.1 : \Append{L}{x}

Ex.2 : \Prepend{L}{y}

Ex.3 : \PopAt{L}{3}
\end{latexex}


% ---------------------- %


\newparaexample{Versions POO}

Voici des notations à la fois concises et aisées à comprendre
\footnote{
	L'opérateur point \POOpoint{} est défini dans la macro \texttt{\textbackslash{}POOpoint}. 
	Ceci permet de personnaliser facilement cet opérateur.
}
avec une syntaxe de type POO
\footnote{
	\og POO \fg{} est l'acronyme de \og Programmation Orientée Objet \fg{}.
}.


\begin{latexex}
Ex.1 : \Append*{L}{x}

Ex.2 : \Prepend*{L}{y}

Ex.3 : \PopAt*{L}{3}
\end{latexex}


% ---------------------- %


\newparaexample{Versions symboliques}

Pour finir il est possible d'utiliser des notations symboliques qui sont très efficaces lorsque l'on rédige les algorithmes à la main
\footnote{
	Rappelons que l'opérateur $\AddList$ est défini dans la macro \texttt{\textbackslash{}AddList}.
}.
Notez au passage qu'ici de petits calculs automatiques facilitent la rédaction et surtout que la macro \macro{PopAt**} prend un argument de plus que les macros \macro{PopAt} et \macro{PopAt*}, cet argument étant pour indiquer où stocker l'élément retiré de la liste.


\begin{latexex}
Ex.1 : \Append**{L}{x}

Ex.2 : \Prepend**{L}{y}

Ex.3 : \PopAt**{e}{L}{3}
\end{latexex}


\begin{remark}
	Voici des points importants à connaître.
	\begin{enumerate}
		\item \verb#\PopAt**{e}{L}{1}# produit \PopAt**{e}{L}{1} sans écrire $\ListUntil{L}{0} \AddList \ListFrom{L}{2}$ puisque pour le package les indices des listes commencent toujours à $1$.

		\smallskip
		\item L'écriture symbolique \PopAt**{e}{L}{k} s'obtient tout simplement via \verb#\PopAt**{e}{L}{k}#.

		\smallskip
		\item Par contre \verb#\PopAt**{e}{L}{k-1}# aboutit à \PopAt**{e}{L}{k-1} ce qui n'est pas joli ! 
	          Dans ce cas, taper \verb#\PopAt**{e}{L}{k-2 | k-1 | k}# aide la macro à produire \PopAt**{e}{L}{k-2 | k-1 | k}.
	\end{enumerate}
\end{remark}


% ---------------------- %


\subsubsection{Modifier une liste en extrayant des sous-listes}

\newparaexample{}

\macro{KeepLR} vient de \og keep left and right \fg{} soit \inenglish{garder à droite et à gauche}.

\begin{latexex}
\KeepLR{L}{a}{b}

\end{latexex}


% ---------------------- %


\newparaexample{}

\begin{latexex}
\KeepL{L}{a}

\KeepR{L}{b}
\end{latexex}

\end{document}

