\documentclass[12pt,a4paper]{article}

\makeatletter
	\usepackage{tnsmath}  % ATTENTION !!!!

% ---------------------- %
% -- GENERAL SETTINGS -- %
% ---------------------- %

\usepackage[
	top    = 2cm,
	bottom = 2cm,
	left   = 1.5cm,
	right  = 1.5cm
]{geometry}

\usepackage[utf8]{inputenc}
\usepackage[T1]{fontenc}
\usepackage{ucs}

\usepackage[french]{babel,varioref}

\usepackage{color}
\usepackage{hyperref}
\hypersetup{
    colorlinks,
    citecolor = black,
    filecolor = black,
    linkcolor = black,
    urlcolor  = black
}
\usepackage[numbered]{bookmark}

\usepackage{enumitem}
\usepackage{multicol}
\usepackage{longtable}
\usepackage{makecell}

\setlength{\parindent}{0cm}
\setlist{noitemsep}



% --------------- %
% -- TOC & Co. -- %
% --------------- %

\usepackage[raggedright]{titlesec}

%\renewcommand\thechapter{\Alph{chapter}.}
\renewcommand\thesection{\Roman{section}.}
\renewcommand\thesubsection{\arabic{subsection}.}
\renewcommand\thesubsubsection{\roman{subsubsection}.}


\titleformat{\paragraph}[hang]{\normalfont\normalsize\bfseries}{\theparagraph}{1em}{}
\titlespacing*{\paragraph}{0pt}{3.25ex plus 1ex minus .2ex}{0.5em}


% Source
%    * https://tex.stackexchange.com/a/558025/6880
\usepackage{tocbasic}[2020/07/22]% needs KOMA-Script version 3.31

\DeclareTOCStyleEntries[
    raggedentrytext,
    linefill = \hfill,
    indent   = 0pt,
    dynindent,
    numwidth = 0pt,
    numsep   = 1ex,
    dynnumwidth
]{tocline}{
	chapter,
	section,
	subsection,
	subsubsection,
	paragraph,
	subparagraph
}

\DeclareTOCStyleEntry[indentfollows = chapter]{tocline}{section}



% ----------- %
% -- TOOLS -- %
% ----------- %

\usepackage{ifplatform}
\usepackage{ifthen}
\usepackage{macroenvsign}
\usepackage{pgffor}



% ------------------------- %
% -- SPECIAL FORMATTINGS -- %
% ------------------------- %

\usepackage{amsthm}

\usepackage{tcolorbox}


% -- LISTINGS -- %

%\tcbuselibrary{listingsutf8}
\tcbuselibrary{minted, breakable}

\newtcblisting{latexex}{%
    breakable,%
    sharp corners,%
    left   = 1mm, right = 1mm,%
    bottom = 1mm, top   = 1mm,%
    %colupper = red!75!blue,%
    listing side text
}

\newtcbinputlisting{\inputlatexex}[2][]{%
    listing file={#2},%
    breakable,
    sharp corners,%
    left   = 1mm, right = 1mm,%
    bottom = 1mm, top   = 1mm,%
    %colupper = red!75!blue,%
    listing side text
}


\newtcblisting{latexex-flat}{%
    breakable,
    sharp corners,%
    left   = 1mm, right = 1mm,%
    bottom = 1mm, top   = 1mm,%
    %colupper = red!75!blue,%
}

\newtcbinputlisting{\inputlatexexflat}[2][]{%
    listing file={#2},%
    breakable,
    sharp corners,%
    left   = 1mm, right = 1mm,%
    bottom = 1mm, top   = 1mm,%
    %colupper = red!75!blue,%
}


\newtcblisting{latexex-alone}{%
    breakable,
    sharp corners,%
    left   = 1mm, right = 1mm,%
    bottom = 1mm, top   = 1mm,%
    %colupper = red!75!blue,%
    listing only
}

\newtcbinputlisting{\inputlatexexalone}[2][]{%
    listing file={#2},%
    breakable,
    sharp corners,%
    left   = 1mm, right = 1mm,%
    bottom = 1mm, top   = 1mm,%
    %colupper = red!75!blue,%
    listing only
}


\newcommand\inputlatexexcodeafter[1]{%
    \begin{center}
        \input{#1}
    \end{center}

    \vspace{-.5em}
    
    Le rendu précédent a été obtenu via le code suivant.
    
    \inputlatexexalone{#1}
}


\newcommand\inputlatexexcodebefore[1]{%
    \inputlatexexalone{#1}
    \vspace{-.75em}
    \begin{center}
        \textit{\footnotesize Rendu du code précédent}
        
        \medskip
        
        \input{#1}
    \end{center}
}


% -- REMARK -- %

\theoremstyle{definition}
\newtheorem*{remark}{Remarque}


% -- EXAMPLE -- %

\newcounter{paraexample}[subsubsection]

\newcommand\@newexample@abstract[2]{%
    \paragraph{%
        #1%
        \if\relax\detokenize{#2}\relax\else {} -- #2\fi%
    }%
}

\newcommand\newparaexample{\@ifstar{\@newparaexample@star}{\@newparaexample@no@star}}

\newcommand\@newparaexample@no@star[1]{%
    \refstepcounter{paraexample}%
    \@newexample@abstract{Exemple \theparaexample}{#1}%
}

\newcommand\@newparaexample@star[1]{%
    \@newexample@abstract{Exemple}{#1}%
}


% -- CHANGE LOG -- %

\newcommand\topic{\@ifstar{\@topic@star}{\@topic@no@star}}

\newcommand\@topic@no@star[1]{%
    \textbf{\textsc{#1}.}%
}

\newcommand\@topic@star[1]{%
    \textbf{\textsc{#1} :}%
}


% -- ABOUT MACROS & Co. -- %

\newcommand\env[1]{\texttt{#1}}
\newcommand\macro[1]{\env{\textbackslash{}#1}}

\newcommand\separation{
    \medskip
    \hfill\rule{0.5\textwidth}{0.75pt}\hfill
    \medskip
}


\newcommand\extraspace{
    \vspace{0.25em}
}


\newcommand\whyprefix[2]{%
    \textbf{\prefix{#1}}-#2%
}

\newcommand\mwhyprefix[2]{%
    \texttt{#1 = #1-#2}%
}

\newcommand\prefix[1]{%
    \texttt{#1}%
}


\newcommand\inenglish{\@ifstar{\@inenglish@star}{\@inenglish@no@star}}

\newcommand\@inenglish@star[1]{%
    \emph{\og #1 \fg}%
}

\newcommand\@inenglish@no@star[1]{%
    \@inenglish@star{#1} en anglais%
}


\newcommand\ascii{\texttt{ASCII}}

	\usepackage{00-keywords}
	\usepackage{01-algo-env}
	\usepackage{03-affectation}
	
	\usepackage{05-list}
\makeatother


\begin{document}

%\section{Des outils pour les algorithmes} \label{algo-extra}

\subsection{Listes}

Avant de commencer il faut savoir que la convention retenue pour la numérotation des indices des listes est de commencer à $1$ \emph{(ceci est plus naturel pour un humain)}.


\subsubsection{Opérations de base.}

\newparaexample{Longueur d'une liste}

\begin{latexex}
$\Len(L)$
\end{latexex}


% ---------------------- %


\newparaexample{Liste vide et liste en extension}

\begin{latexex}
$\EmptyList$

$\List{4 ; 7 ; -1}$
\end{latexex}


% ---------------------- %


\newparaexample{Extraire certains éléments}

\begin{latexex}
k\ieme{} élément :
$\ListElt{L}{k}$

Du 1\ier{} au k\ieme{} :
$\ListUntil{L}{k}$

À partir du k\ieme{} :
$\ListFrom{L}{k}$
\end{latexex}


% ---------------------- %


\newparaexample{Concaténer des listes}

\begin{latexex}
$L =          \ListUntil{L}{k-1}
     \AddList \ListElt{L}{k}
     \AddList \ListFrom{L}{k+1}$
\end{latexex}


% ---------------------- %


\subsubsection{Modifier une liste avec un élément}

\newparaexample{Versions textuelles}

\prefix{append} et \prefix{prepend} signifient \inenglish*{ajouter} et \inenglish{préfixer}.
Quant à \prefix{pop}, il signifie \inenglish*{éclater}.


\begin{latexex}
Ex.1 : \Append{L}{x}

Ex.2 : \Prepend{L}{y}

Ex.3 : \PopAt{L}{3}
\end{latexex}


% ---------------------- %


\newparaexample{Versions POO}

Voici des notations à la fois concises et aisées à comprendre
\footnote{
	L'opérateur point \POOpoint{} est défini dans la macro \texttt{\textbackslash{}POOpoint}. 
	Ceci permet de personnaliser facilement cet opérateur.
}
avec une syntaxe de type POO
\footnote{
	\og POO \fg{} est l'acronyme de \og Programmation Orientée Objet \fg{}.
}.


\begin{latexex}
Ex.1 : \Append*{L}{x}

Ex.2 : \Prepend*{L}{y}

Ex.3 : \PopAt*{L}{3}
\end{latexex}


% ---------------------- %


\newparaexample{Versions symboliques}

Pour finir il est possible d'utiliser des notations symboliques qui sont très efficaces lorsque l'on rédige les algorithmes à la main
\footnote{
	Rappelons que l'opérateur $\AddList$ est défini dans la macro \texttt{\textbackslash{}AddList}.
}.
Notez au passage qu'ici de petits calculs automatiques facilitent la rédaction et surtout que la macro \macro{PopAt**} prend un argument de plus que les macros \macro{PopAt} et \macro{PopAt*}, cet argument étant pour indiquer où stocker l'élément retiré de la liste.


\begin{latexex}
Ex.1 : \Append**{L}{x}

Ex.2 : \Prepend**{L}{y}

Ex.3 : \PopAt**{e}{L}{3}
\end{latexex}


\begin{remark}
	Voici des points importants à connaître.
	\begin{enumerate}
		\item \verb#\PopAt**{e}{L}{1}# produit \PopAt**{e}{L}{1} sans écrire $\ListUntil{L}{0} \AddList \ListFrom{L}{2}$ puisque pour le package les indices des listes commencent toujours à $1$.

		\smallskip
		\item L'écriture symbolique \PopAt**{e}{L}{k} s'obtient tout simplement via \verb#\PopAt**{e}{L}{k}#.

		\smallskip
		\item Par contre \verb#\PopAt**{e}{L}{k-1}# aboutit à \PopAt**{e}{L}{k-1} ce qui n'est pas joli ! 
	          Dans ce cas, taper \verb#\PopAt**{e}{L}{k-2 | k-1 | k}# aide la macro à produire \PopAt**{e}{L}{k-2 | k-1 | k}.
	\end{enumerate}
\end{remark}


% ---------------------- %


\subsubsection{Modifier une liste en extrayant des sous-listes}

\newparaexample{}

\macro{KeepLR} vient de \og keep left and right \fg{} soit \inenglish{garder à droite et à gauche}.

\begin{latexex}
\KeepLR{L}{a}{b}

\end{latexex}


% ---------------------- %


\newparaexample{}

\begin{latexex}
\KeepL{L}{a}

\KeepR{L}{b}
\end{latexex}

\end{document}

