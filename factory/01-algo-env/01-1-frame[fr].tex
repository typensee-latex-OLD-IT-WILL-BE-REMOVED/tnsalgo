\documentclass[12pt,a4paper]{article}

\makeatletter
	\input{../config/header[fr].sty}
	\usepackage{00-keywords}
	% == PACKAGES USED == %

\RequirePackage{mathtools}


% == DEFINITIONS == %

% Source: https://tex.stackexchange.com/a/510011/6880

\newcommand\PutIn{\rightarrow}
\newcommand\MPutIn{\rightrightarrows}

\newcommand\MStore{\leftleftarrows}


\newcommand{\tnsalgo@special@hat}[2]{%
    \begingroup
        \sbox\z@{$\m@th#1=$}%
        \ooalign{%
            \hidewidth\raisebox{-0.3\ht\z@}{$\m@th#1\widehat{}$}\hidewidth\cr
            \box\z@\cr
        }%
    \endgroup
}

\newcommand\Store{\@ifstar{\@Store@pre@star}{\@Store@no@star}}
\newcommand\@Store@pre@star{\@ifstar{\@Store@star@star}{\@Store@star}}

\newcommand\@Store@no@star{\leftarrow}
\newcommand\@Store@star{\mathrel{\mathpalette\tnsalgo@special@hat\relax}}
\newcommand\@Store@star@star{\triangleq}


	% == PACKAGES USED == %

\RequirePackage{amsmath}


% == DEFINITIONS == %

% -- FOR-RANGE LOOP -- %

\newcommand\ForRange{\@ifstar{\@ForRange@pre@star}{\@ForRange@no@star}}
\newcommand\@ForRange@pre@star{\@ifstar{\@ForRange@star@star}{\@ForRange@star}}

\newcommand\@ForRange@no@star[4]{
	\For{\text{$#1$} \ComingFrom \text{$#2$} \GoingTo \text{$#3$}}{#4}
}

\newcommand\@ForRange@star[4]{
	\For{\text{$#1$} \From \text{$#2$} \To \text{$#3$}}{#4}
}

\newcommand\@ForRange@star@star[4]{
	\For{$#1 \in \CSinterval{#2}{#3}$}{#4}
}


	\usepackage{01-algo-env}
\makeatother



\begin{document}

%\section{Algorithmes en language naturel}

\subsection{L'environnement \texttt{algo}}

\subsubsection{Cadre et largeur}

L'option \verb#frame# de l'environnement \verb#algo# demande d'encadrer les algorithmes afin de les rendre plus visibles.
Indiquons qu'ici sont utilisées les macros \macro{NNs} et \macro{dsum} fournies par le package \verb#tnsmath#.

\inputlatexex{examples/algo-natural-stupid.extra.tex}


L'environnement \verb#algo# propose l'option \verb#scale# pour indiquer la largeur relativement à celle de la ligne et une autre \verb#center# pour centrer l'algorithme qui permettent d'obtenir ce qui suit.

\vspace{-1em}
\inputlatexexcodeafter{examples/algo-natural-stupid-scaled-center.extra.tex}

\end{document}