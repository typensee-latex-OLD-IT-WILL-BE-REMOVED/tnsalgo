\documentclass[12pt,a4paper]{article}

\makeatletter
	\input{../config/header[fr].sty}
	% == PACKAGES USED == %

\RequirePackage[french, vlined]{algorithm2e}
\RequirePackage{tcolorbox}

\RequirePackage{simplekv}


% == DEFINITIONS == %

% -- ENVIRONMENT -- %

\setKVdefault[tnspalgo@algo@env@keys]{%
    frame   = false,
    center  = false,
    notitle = false,
    title   = {},
    scale   = 1
}


\newenvironment{algo}[1][]{
    \useKVdefault[tnspalgo@algo@env@keys]%
    \setKV[tnspalgo@algo@env@keys]{#1}%
    \edef\thisscale{\useKV[tnspalgo@algo@env@keys]{scale}}%
    \edef\thistitle{\useKV[tnspalgo@algo@env@keys]{title}}%
    %
    \ifboolKV[tnspalgo@algo@env@keys]{center}{\centering}{}
    %
    \ifboolKV[tnspalgo@algo@env@keys]{frame}{%
    	\begin{tcolorbox}[
    		colback = white,
    		width   = \thisscale\linewidth,
    		breakable
    	]
			\vspace{-0.25em}
	}{}
			\begin{algorithm}[H]
	% Title ?
	\ifboolKV[tnspalgo@algo@env@keys]{notitle}{}{
	% Empty tile
    	\if\relax\thistitle\relax
    		\tnspalgo@void@caption{}
	% None empty tile
    	\else
    		\caption{\thistitle}
    	\fi
	}
}{
			\end{algorithm}
	%
    \ifboolKV[tnspalgo@algo@env@keys]{frame}{%
			\vspace{-0.5em}
		\end{tcolorbox}
	}{}
}



% -- CAPTIONS -- %

% Source
%	* https://tex.stackovernet.com/fr/q/66875#214011
%	* https://tex.stackexchange.com/a/510498/6880

\renewcommand{\@algocf@capt@plain}{above}
\renewcommand{\algocf@caption@plain}{\box\algocf@capbox\vskip\AlCapSkip}%

\setlength{\AlCapSkip}{.1em}


\newcommand\tnspalgo@void@caption{
	\SetAlgoCaptionSeparator{}	% No separator (default colon)
	\SetAlCapNameSty{}			% No caption text
	\caption{}
}

	
	\usepackage{03-affectation}
\makeatother


\begin{document}

%\section{Algorithmes en language naturel}

\subsection{Affectations simples ou multiples}

\subsubsection{Affectation simple avec une flèche}

\begin{latexex}
$x \Store 3$ ou
$3 \PutIn x$
\end{latexex}


% ---------------------- %


\subsubsection{Affectation simple avec un signe égal décoré}

Deux notations alternatives existent
\footnote{
	La 2\ieme{} notation vient du langage B qui permet de spécifier et prouver des programmes.
}.
Voici comment les obtenir.

\begin{latexex}
$x \Store* 3$ ou
$x \Store** 3$
\end{latexex}


% ---------------------- %


\subsubsection{Affectations multiples en parallèle}

\begin{latexex}
$a, b, c \MStore x, y, z$ ou
$x, y, z \MPutIn a, b, c$
\end{latexex}


\begin{remark}
	La multi-affectation se faisant en parallèle, le résultat de $a, b, c \MStore 2, a + b, c - b$ ne sera pas semblable à celui de $a \Store 2$ suivi de $b \Store a + b$ puis de $c \Store c - b$ car dans le second cas les variables $a$ et $b$ évoluent avant de nouvelles affectations simples. En fait la multi-affectation précédente correspond aux actions suivantes.

    \begin{algo}[
    	frame,
    	title = {Comment $a, b, c \MStore 2, a + b, c - b$ fonctionne-t-il ?}
    ]
    
    	$a_{memo} \Store a$ ;
    	$b_{memo} \Store b$ ;
    	$c_{memo} \Store c$
    	\\
    	\BlankLine
    	$a \Store 2$
    	\\
    	$b \Store a_{memo} + b_{memo}$
    	\\
    	$c \Store c_{memo} - b_{memo}$
    \end{algo}
\end{remark}


% ---------------------- %


\subsection{Fiches techniques}

\IDope{Store  }

\IDope{Store* }

\IDope{Store**}

\extraspace

\IDope{PutIn }

\extraspace

\IDope{MStore}

\IDope{MPutIn}

\end{document}

