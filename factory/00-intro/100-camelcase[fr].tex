\documentclass[12pt,a4paper]{article}

\makeatletter
	\input{../config/header[fr].sty}
\makeatother


\begin{document}

\section{Noms des macros}

%\subsection{Algorithmes en langage naturel}

Pour les algorithmes en langage naturel, il est fait appel au package \verb#algorithm2e# qui utilise, et abuse
\footnote{
	Ce type de convention est un peu pénible lors de la saisie au clavier.
},
de la notation dite en bosses de dromadaire
\footnote{
	 On parle aussi de casse à la Pascal en référence au langage de programmation.
}
comme par exemple avec \macro{BlankLine} et \macro{ElseIf} au lieu de \macro{blankline} et \macro{elseif}.
Par souci de cohérence les nouvelles macros ajoutées par \verb#tnsalgo# en lien avec les algorithmes utilisent aussi cette convention même si par exemple l'auteur aurait préféré proposer \macro{putin} et \macro{forrange} à la place de \macro{PutIn} et \macro{ForRange}.


% ---------------------- %


%\subsection{Ordinogrammes} Utile ? Si oui parler de tkz-tab ==> camelCase

\end{document}

