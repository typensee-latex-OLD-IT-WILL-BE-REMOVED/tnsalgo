\documentclass[12pt,a4paper]{article}

\makeatletter
    \input{../config/header[fr].sty}

    \usepackage{01-flowchart}
\makeatother


\newcommand\Store{\leftarrow}
\newcommand\List[1]{[\,#1\,]}
\newcommand\EmptyList{\List{}}


\begin{document}

%\section{Ordinogrammes}

\subsection{L'environnement \texttt{algochart}}

Tous les codes seront placés dans l'environnement \verb+algochart+ qui pour le moment est juste un alias de l'environnement \verb+tikzpicture+ proposé par \verb+TikZ+ qui fait le principal du travail
\footnote{
	\texttt{algochart} vient de la contraction de \og algorithmic \fg{} et \og flowchart \fg{} soit \og algotithmique \fg{} et \og diagramme \fg{} en anglais.
}.
\verb#tnsalgo# définit juste quelques styles et quelques macros pour faciliter la saisie des ordinogrammes afin de travailler efficacement avec les macros \macro{node} et \macro{path} proposées par \verb#TikZ#.

\end{document}
