\documentclass[12pt,a4paper]{article}

\makeatletter
    \input{../config/header[fr].sty}

    \usepackage{01-flowchart}
\makeatother


\newcommand\Store{\leftarrow}
\newcommand\List[1]{[\,#1\,]}
\newcommand\EmptyList{\List{}}


\begin{document}

\newpage
\section{Ordinogrammes}

\subsection{C'est quoi un ordinogramme} \label{tnsalgo-flowchart-firstexa}

Les ordinogrammes
\footnote{
    Le mot \og ordinogramme \fg{} vient des mots \og ordinateur \fg{}, du latin \og ordinare \fg{} soit \og mettre en ordre \fg{}, et du grec ancien \og gramma \fg{} soit \og lettre, écriture \fg{}.
}
sont des diagrammes que l'on peut utiliser pour expliquer des algorithmes très simples
\footnote{
    Cet outil pédagogique montre très vite ses limites. Essayez par exemple de tracer un ordinogramme pour expliquer comment résoudre une équation du 2\ieme{} degré.
}.

\medskip


Voici un exemple expliquant comment résoudre dans $\RR$ l'équation en $x$ $a x^2 + b = 0$ lorsque $a \neq 0$ et $b \neq 0$ : le code utilisé est donné plus tard dans la section \ref{tnsalgo-flowchart-firstexa-code} \emph{(ce code sera très aisé à comprendre une fois lues les sections à venir)}.

\begin{center}
    \small
    \input{examples/ac-merly-2nd-degree.tkz}
\end{center}

\end{document}
