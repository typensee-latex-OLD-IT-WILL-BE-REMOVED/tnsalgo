\documentclass[12pt,a4paper]{article}

\makeatletter
	\input{../config/header[fr].sty}
\makeatother


\begin{document}

%\section{Algorithmes en langage naturel}

%\subsection{L'environnement \texttt{algo}}

\subsection{Fiches techniques}

\IDenv[n]{algochart}

Cet environnement s'utilise avec les styles \verb#TikZ# suivants.

\begin{enumerate}
	\item \verb#acio# pour un noeud indiquant l'entrée ou la sortie de l'algorithme.

	\item \verb#acinstr# pour un noeud indiquant une instruction.

	\item \verb#acif# pour un noeud indiquant un test.

	\smallskip

	\item \verb#aclink# pour un chemin liant deux noeuds.

	\smallskip

	\item \verb#acbackloopleft# à utiliser avec \verb#to# pour une boucle de rétroaction par la gauche entre deux noeuds.
	      Ce style peut recevoir l'argument optionnel \verb#x# permet d'indiquer le décalage vers la gauche.
		  Par défaut \verb#x = 5em#.

	\item \verb#acbackloopright# à utiliser avec \verb#to# pour une boucle de rétroaction par la droite entre deux noeuds.
	      Ce style peut recevoir l'argument optionnel \verb#x# permet d'indiquer le décalage vers la droite.
		  Par défaut \verb#x = 5em#.

	\smallskip

	\item \verb#aczigzag# à utiliser avec \verb#to# pour un chemin \fg cassé à un angle droit \fg.
	      Ce style peut recevoir les arguments optionnels suivants.
	      \begin{enumerate}
	      		\item \verb#xstart# permet d'indiquer un décalage horizontal au début.
			          Par défaut \verb#xstart = 0mm#.

	      		\item \verb#x# permet d'indiquer un décalage horizontal à la fin.
			          Par défaut \verb#x = 3mm#.

	      		\item \verb#y# permet d'indiquer un décalage vertical à la fin.
			          Par défaut \verb#y = 5mm#.
	      \end{enumerate}
\end{enumerate}


% ---------------------- %


\separation


\IDmacro[a]{aclabelabove}{1}

\IDmacro[a]{aclabelbelow}{1}

\extraspace

\IDmacro[a]{aclabelleft}{1}

\IDmacro[a]{aclabelright}{1}


\IDarg{} le texte au début du chemin.


% ---------------------- %


\separation


\IDope{acusebw}      \hfill  \prefix{bw = b-lack \& w-hite}

\IDope{acusecolor}

\end{document}
